%!TEX TS-program = xelatex
\documentclass[a4paper,french,11pt]{article}
\usepackage{geometry}
\geometry{verbose,letterpaper,tmargin=3cm,bmargin=3cm,lmargin=3cm,rmargin=3cm}

% XeTex related packages
\usepackage{fontspec,xltxtra,xunicode}

\usepackage[breaklinks=true]{hyperref}
\usepackage[french]{babel}
\usepackage[most]{tcolorbox}
\usepackage{amsmath}
\usepackage{array}
\usepackage{booktabs}
\usepackage{colortbl}
\usepackage{fancyhdr}
\usepackage{framed}
\usepackage{graphicx}
\usepackage{longtable}
\usepackage{mdframed}
\usepackage{minted}
\usepackage{titlesec}
\usepackage{titling}

% -- Pandoc latex config: https://gist.github.com/1017790
% Paragraph format (European style)
\setlength{\parindent}{0pt}
\setlength{\parskip}{6pt plus 2pt minus 1pt}

% Lists formatting
% -- Redefine labelwidth for lists; otherwise, the enumerate package will cause
% -- markers to extend beyond the left margin.
\makeatletter\AtBeginDocument{
  \renewcommand{\@listi}
    {\setlength{\labelwidth}{4em}}
}\makeatother
\usepackage{enumerate}

\providecommand{\tightlist}{%
  \setlength{\itemsep}{0pt}\setlength{\parskip}{0pt}}


% Table formatting
% -- This is needed because raggedright in table elements redefines \\:
\newcommand{\PreserveBackslash}[1]{\let\temp=\\#1\let\\=\temp}
\let\PBS=\PreserveBackslash

% Subscripts
\newcommand{\textsubscr}[1]{\ensuremath{_{\scriptsize\textrm{#1}}}}

% Web-style links
\hypersetup{
    colorlinks,
    citecolor=blue,
    filecolor=blue,
    linkcolor=blue,
    urlcolor=blue
}

\setminted{
    breaklines=true,
    samepage=true
}

\AtBeginEnvironment{minted}{
  \renewcommand{\fcolorbox}[4][]{#4}
}

% -- Custom GCC related config
% Reduce minted space
\setlength\partopsep{-\topsep}
\addtolength\partopsep{-\parskip}
\addtolength\partopsep{0.3cm}

% Reduce title top margin
\setlength{\droptitle}{-5em}

% Header/footer
\setlength{\headheight}{15pt}
\pagestyle{fancy}
\fancyhf{}
\rhead{GCC! }
\lhead{}
\cfoot{\thepage}
\fancyfoot[L]{\includegraphics[width=2cm]{../logo_gcc_court.pdf}}
\fancyfoot[R]{\includegraphics[width=1.2cm]{../prologin_cube_bw.pdf}}

% Custom color environments
\definecolor{grey}{RGB}{240, 240, 240}
\setminted[text]{bgcolor=grey}
\setminted[python]{linenos=true,frame=single,framesep=4pt}
\setminted[html]{linenos=true,frame=single,framesep=4pt}
\setminted[css]{linenos=true,frame=single,framesep=4pt}
\setminted[js]{linenos=true,frame=single,framesep=4pt}
\newenvironment{exercise}{\begin{mdframed}[nobreak=true]}{\end{mdframed}}

% Remove section numbering
\makeatletter
\renewcommand{\@seccntformat}[1]{}
\def\maxwidth{\ifdim\Gin@nat@width>\linewidth\linewidth\else\Gin@nat@width\fi}
\def\maxheight{\ifdim\Gin@nat@height>\textheight\textheight\else\Gin@nat@height\fi}
\makeatother
% Scale images if necessary, so that they will not overflow the page
% margins by default, and it is still possible to overwrite the defaults
% using explicit options in \includegraphics[width, height, ...]{}
\setkeys{Gin}{width=\maxwidth,height=\maxheight,keepaspectratio}

\title{
    \protect\centering\protect\includegraphics[height=3cm]{../logo_gcc_long.pdf}\\
    {}
}
\author{GCC! \textendash{} Prologin}
\date{}

\begin{document}

\maketitle

\hypertarget{avant-de-commencer}{%
\section{Avant de commencer}\label{avant-de-commencer}}

\begin{itemize}
\item
  Si tu ne comprends pas quelque chose, n'hésite surtout pas à demander
  à un orga ! Nous sommes là pour ça !
\item
  On te conseille de créer un fichier par exercice, tous dans le même
  dossier.
\item
  On te conseille fortement de tester tous les codes présents dans ce
  TP, même s'ils ne sont que des exemples. Si tu ne comprends pas
  comment ils fonctionnent, ou que quelque chose cloche, n'hésite pas à
  appeler un orga.
\end{itemize}

\hypertarget{la-commande-print}{%
\section{La commande print}\label{la-commande-print}}

D'habitude, le tout premier programme qu'on écrit quand on commence à
apprendre un langage de programmation sert à afficher ``Hello world !''
à l'écran. On va commencer par là nous aussi. Ouvre un nouveau fichier
et enregistre-le sous le nom \mintinline[]{text}{hello_world.py}
\textgreater{} On évite de mettre des espaces dans les noms de fichier.
On les remplace par le symbole underscore \mintinline[]{text}{_}. On
évite aussi les majuscules.

\begin{quote}
Le ``.py'' s'appelle une extension. Cela sert à préciser à l'ordinateur
que ce qui est écrit dans le fichier est écrit dans le langage python,
ce qui peut permettre à certains éditeurs de colorer les mots clefs de
Python.
\end{quote}

Dans ce fichier, écris

\begin{minted}[autogobble]{python}
print("Hello world !")
\end{minted}

et enregistre.

Maintenant, nous allons exécuter ce programme. Pour cela, il suffit de
cliquer sur ``Run'' et tu verras l'écran se diviser en deux. Sur la
partie basse, s'affiche le résultat de ton programme.

Comme on le voulait, la phrase Hello world ! s'affiche. Maintenant,
l'exécution de notre programme est terminée.

\hypertarget{duxe9couverte-des-erreurs}{%
\section{Découverte des erreurs}\label{duxe9couverte-des-erreurs}}

Pour donner des instructions à l'ordinateur en utilisant Python, on doit
utiliser un langage précis et codifié. Observons notre première
instruction : le message qu'on veut écrire doit être entouré de
guillemets et être entre parenthèses.

Essaie d'enlever les parenthèses ou les guillemets et d'exécuter à
nouveau le programme (n'oublie pas d'enregistrer quand tu as fini de
modifier !).

Comme tu peux le constater, au lieu d'afficher Hello world ! Python
renvoie ce qu'on appelle une erreur : c'est un message pour expliquer
que le programme a un problème. Le plus souvent, il précise la nature de
l'erreur et la ligne où il l'a détectée. Mais attention, ce n'est pas
forcément à cette ligne qu'on devra modifier quelque chose pour que le
programme fonctionne.

\hypertarget{exercice}{%
\subsection{Exercice :}\label{exercice}}

Écris un programme qui affiche la phrase :

\begin{minted}[autogobble]{text}
Bienvenue au stage d’informatique GCC! !
\end{minted}

Exécute-le pour vérifier qu'il fonctionne.

\hypertarget{variables}{%
\section{Variables}\label{variables}}

Nous allons maintenant demander à Python de retenir des valeurs pour
nous et de faire des opérations avec ces valeurs. Pour cela, on va
utiliser ce qui s'appelle une variable.

Quand on utilise Python, il faut s'imaginer qu'on a accès à un grand
meuble avec plein de tiroirs. Ces tiroirs sont les variables. Sur chaque
tiroir il y a une étiquette : c'est le nom de la variable.

Dans un tiroir, on peut mettre une valeur. On peut aussi ouvrir le
tiroir pour lire la valeur qu'on a mise à l'intérieur. Ou encore, on
peut ouvrir le tiroir et remplacer la valeur qu'on y a mise par une
autre.

Pour créer une variable avec Python, on écrit par exemple :

\begin{minted}[autogobble]{python}
x = 2
\end{minted}

Cela veut dire ``ouvre le tiroir avec l'étiquette x et range la valeur 2
à l'intérieur''.

Crée un nouveau fichier \mintinline[]{text}{variable.py} et entre cette
ligne de code, puis exécute-le.

Il ne se passe rien ! C'est normal, on n'a pas demandé à Python
d'afficher quoi que ce soit. Rajoute maintenant
\mintinline[]{text}{print(x)}

et exécute à nouveau le programme. Tu vois que Python affiche la valeur
contenue dans la variable x, ici 2.

\hypertarget{opuxe9rations}{%
\section{Opérations}\label{opuxe9rations}}

Python nous permet d'effectuer des opérations, comme une calculatrice
très perfectionnée. Essaie le programme suivant :

\begin{minted}[autogobble]{python}
print(4+3)
\end{minted}

\begin{quote}
Tu remarques qu'ici, il n'y a pas de guillemets, comme avant lorsqu'on a
écrit print(x). C'est parce qu'on ne met des guillemets que lorsqu'on
veut afficher une chaîne de caractères. Tu peux essayer de voir ce qui
se passe si tu écris plutôt \mintinline[]{text}{print(«x»)} ou
\mintinline[]{text}{print(«4+3»)}.
\end{quote}

Mais ce qui est encore plus pratique, c'est qu'on peut faire des
opérations avec les valeurs contenues dans nos variables. Essaie par
exemple le programme suivant :

\begin{minted}[autogobble]{python}
a = 4
b = 3
resultat = a+b
print(resultat)
\end{minted}

\begin{quote}
À quoi ça sert ? Ici, le calcul est très simple, mais imagine que tu
aies beaucoup d'opérations à effectuer. En écrivant ton programme avec
des variables, tu n'as qu'à changer leur valeur au début pour pouvoir
refaire toutes ces opérations sans avoir à écrire de nouveau du code. Si
tu voulais plutôt calculer 5+3 ici, il suffirait de remplacer la
première ligne par a=5. Cela va devenir encore plus clair avec le
prochain paragraphe, où tu verras qu'on peut demander à l'utilisateur un
nombre de son choix qu'on mettra ensuite dans une variable.
\end{quote}

\begin{quote}
Il s'agit de ton premier programme avec plusieurs instructions, et donc
plusieurs lignes, car en python, on met une instruction par ligne. Une
instruction, c'est une tâche qu'on demande à l'ordinateur d'effectuer :
affiche ceci, mets cette valeur dans cette variable, etc.
\end{quote}

Mais Python ne permet pas seulement de faire des additions. Tu peux
aussi utiliser les signes \mintinline[]{text}{-},
\mintinline[]{text}{*}, \mintinline[]{text}{/} pour effectuer des
soustractions, des multiplications et des divisions. N'hésite pas à
essayer !

\hypertarget{pour-aller-plus-loin}.

Essaie-les pour comprendre ce qu'ils font.

\hypertarget{exercice-1}{%
\subsection{Exercice :}\label{exercice-1}}

Voici un programme python :

\begin{minted}[autogobble]{python}
a = 7
b = 3
c = a*2
d = c+b
print(d-4)
\end{minted}

Que va afficher l'ordinateur à la fin de ce programme ? Vérifie en
l'essayant.

\hypertarget{exercice-2}{%
\subsection{Exercice :}\label{exercice-2}}

Voici un programme python :

\begin{minted}[autogobble]{python}
x = 3
x = x – 2
print(x)
\end{minted}

Que va afficher l'ordinateur à la fin de ce programme ? Vérifie en
l'essayant.

\hypertarget{exercice-3}{%
\subsection{Exercice :}\label{exercice-3}}

On cherche à écrire un programme python qui inverse le contenu de deux
variables. On propose le programme suivant :

\begin{minted}[autogobble]{python}
a = 4
b = 6
a = b
b = a
print("a vaut" + str(a))
print("b vaut" + str(b))
\end{minted}

\begin{enumerate}
\def\labelenumi{\arabic{enumi}.}
\tightlist
\item
  À ton avis, que va afficher l'ordinateur ? Essaie pour vérifier.
\item
  Écris un programme qui permet d'inverser le contenu de deux variables.
\end{enumerate}

\begin{quote}
Pour afficher une chaîne de caractères (du texte, entre guillemets) et
un nombre dans la même phrase, on va devoir dire à l'ordinateur de
traiter le nombre comme du texte. C'est ce que veut dire str(\ldots{})
Le signe + entre deux morceaux de texte veut dire qu'on va les mettre
l'une à la suite de l'autre. Cette opération s'appelle concaténer.
\end{quote}

\hypertarget{demander-une-valeur-uxe0-lutilisateur}{%
\section{Demander une valeur à
l'utilisateur}\label{demander-une-valeur-uxe0-lutilisateur}}

Pour demander une valeur à l'utilisateur de notre programme, on va
utiliser l'instruction \mintinline[]{text}{input}. Cette instruction dit
à l'ordinateur de suspendre le programme jusqu'à ce que l'utilisateur
entre une réponse et appuie sur entrée.

Nous allons essayer le programme suivant :

\begin{minted}[autogobble]{python}
print("Comment t’appelles-tu ?")
nom = input()
print("Tu t’appelles" + nom)
\end{minted}

Maintenant, nous allons demander à l'utilisateur d'entrer une valeur
chiffrée. Par exemple, demandons-lui son âge. Pour cela, plutôt que
d'écrire \mintinline[]{text}{nom = input()}, on va écrire
\mintinline[]{text}{age = int(input())}. Quand on demande un nombre à
l'utilisateur, il faut utiliser \mintinline[]{text}{int(input())} et pas
seulement \mintinline[]{text}{input()}.

\hypertarget{pour-aller-plus-loin-pourquoi-ce-int}{%
\subsection{\texorpdfstring{Pour aller plus loin : Pourquoi ce
\mintinline[]{text}{int()}
?}{Pour aller plus loin : Pourquoi ce  ?}}\label{pour-aller-plus-loin-pourquoi-ce-int}}

python ne traite pas de la même manière les nombres entiers, les nombres
à virgule, et le texte. Quand \mintinline[]{text}{input()} récupère
l'entrée de l'utilisateur, il la traite comme une chaîne de texte. Or,
on ne peut pas faire certaines opérations sur les chaînes de texte
(soustraction, division, etc.) dont on aurait besoin. Le
\mintinline[]{text}{int()} dit à l'ordinateur qu'en fait, on veut
traiter ce que l'utilisateur entre comme un nombre. La chaîne de texte
``4'' devient ainsi le nombre 4. C'est le même mécanisme dans l'autre
sens qu'on a vu avec \mintinline[]{text}{str()} avant.

Le programme sera alors :

\begin{minted}[autogobble]{python}
print("Quel est ton âge ?")
age = int(input())
print("Tu as " + str(age) + " ans.")
\end{minted}

N'oublie pas qu'une fois la valeur entrée dans une variable, on peut
faire des opérations avec elle, et pas seulement l'afficher.

\begin{quote}
Nous allons maintenant faire quelques exercices pour manipuler les
variables ainsi que les instructions print et input. Si tu as
l'impression que les petits programmes que nous écrivons pour l'instant
ne servent pas à grand-chose, ne t'inquiète pas : c'est normal ! Pour
l'instant, on essaie simplement de bien comprendre comment tout
fonctionne, mais tu verras qu'à la fin du stage, ou sur les exercices
bonus de ce TP, tu seras capable d'écrire des programmes qui font des
tâches très complexes et qui te simplifieront la vie ou te permettront
de faire des choses impossibles sans l'informatique.
\end{quote}

\hypertarget{exercice-4}{%
\subsection{Exercice :}\label{exercice-4}}

Écris un programme qui pose trois questions à l'utilisateur : son jour
de naissance, son mois de naissance et son année de naissance. On devra
attendre que l'utilisateur ait répondu pour passer à la question
suivante. Quand on aura les trois réponses, le programme doit alors
afficher : \mintinline[]{text}{Vous êtes né le jour mois année}

\hypertarget{exercice-5}{%
\subsection{Exercice :}\label{exercice-5}}

Écris un programme qui demande son âge à un utilisateur et qui affiche
ensuite: \mintinline[]{text}{Vous aurez 18 ans dans … années}

\begin{quote}
Et si l'utilisateur rentre un nombre supérieur à 18 ? Essaie ! Ce n'est
sans doute pas comme ça qu'on voudrait que le programme se comporte. On
aimerait plutôt qu'il affiche ``vous avez déjà 18 ans'' ou quelque chose
comme ça. C'est justement ce qu'on va apprendre à faire tout de suite !
\end{quote}

\hypertarget{les-conditionnelles-if-else-elif}{%
\section{Les conditionnelles (if -- else --
elif)}\label{les-conditionnelles-if-else-elif}}

Jusqu'à maintenant, nos programmes se sont toujours déroulés en
exécutant ligne par ligne ce qu'on leur demandait. D'abord, ils
faisaient l'instruction donnée à la première ligne, puis celle de la
seconde, etc. jusqu'à la dernière ligne.

Mais parfois, on voudrait pouvoir écrire un morceau de programme qui
n'est exécuté que si une certaine condition est remplie, et un autre si
elle n'est pas remplie. Par exemple, pour reprendre l'exercice
précédent, on voudrait faire quelque chose de différent selon que
l'utilisateur a déjà 18 ans, ou pas encore.

Pour cela, on va utiliser ce qu'on appelle une conditionnelle. En voici
un exemple :

\begin{minted}[autogobble]{python}
print("Quel est ton âge ?")
age = int(input())
if age < 18 :
    resultat = 18-age
    print("Vous aurez 18 ans dans " + resultat + " ans.")
else :
    print("Vous avez déjà 18 ans.")
\end{minted}

\mintinline[]{text}{if} est un mot anglais qui veut dire ``si'' et
\mintinline[]{text}{else} ``sinon''. Ce programme dit donc à
l'ordinateur :

\begin{itemize}
\item
  Si l'âge est inférieur à 18 ans : calcule le nombre d'années avant
  d'avoir 18 ans et mets-le dans la variable résultat, puis affiche
  ``Vous aurez 18 ans dans \mintinline[]{text}{resultat} ans.''
\item
  Sinon, affiche ``Vous avez déjà 18 ans.''
\end{itemize}

\begin{quote}
Tu remarques que ce code a un aspect bien particulier. Il y a des
\mintinline[]{text}{:} à la fin des lignes avec un if ou un else, et les
lignes après sont indentées (on utilise la touche tab pour cela). Il
faut respecter cela pour que le programme fonctionne.
\end{quote}

Pour écrire des conditionnelles, nous allons avoir besoin de savoir
écrire des conditions. On les écrit globalement comme en mathématiques :

\begin{itemize}
\item
  Pour dire ``a est inférieur à b'' on écrit \mintinline[]{text}{a < b}
\item
  Pour dire ``a est supérieur à b'' on écrit \mintinline[]{text}{a > b}
\item
  Pour dire ``a est inférieur ou égal à b'' on écrit
  \mintinline[]{text}{a <= b}
\item
  Pour dire ``a est supérieur ou égal à b'' on écrit
  \mintinline[]{text}{a >= b}
\item
  Pour dire ``a est égal à b'' on écrit \mintinline[]{text}{a == b}
\item
  Pour dire ``a est différent de b'' on écrit \mintinline[]{text}{a!=b}
\end{itemize}

\hypertarget{exercice-6}{%
\subsection{Exercice :}\label{exercice-6}}

Imaginons que nous avons un jeu auquel on gagne une fois qu'on a un
score de 21 ou plus. Écris un programme qui demande son score à
l'utilisatrice puis affiche selon le cas ``Tu as gagné'' ou ``Tu as
perdu''.

\hypertarget{exercice-7}{%
\subsection{Exercice :}\label{exercice-7}}

Cet exercice va nous permettre de mieux comprendre à quoi servent les
tabulations.

Pour chaque question, essaie de trouver ce que le programme python donné
va afficher si on entre les nombres donnés, puis écris-le et exécute-le.
Fais attention, quand tu le recopies, à bien respecter la même
indentation que dans l'énoncé.

\begin{enumerate}
\def\labelenumi{\arabic{enumi}.}
\item
\end{enumerate}

\begin{minted}[autogobble]{python}
print("Quel est ton score ?")
score = int(input())
if score ≥ 100:
    print("Bravo, tu as gagné !")
else :
    print("Tu as perdu.")
print("Merci d’avoir joué !")
\end{minted}

Que se passe-t-il si on entre 50 ? Et si on entre 120 ?

\begin{enumerate}
\def\labelenumi{\arabic{enumi}.}
\setcounter{enumi}{1}
\item
\end{enumerate}

\begin{minted}[autogobble]{python}
print("À combien de stages GCC! as-tu participé ?")
nombre_stages = int(input)
if nombre_stages == 0:
    print("C’est bizarre pourtant, puisque tu es là !")
else:
    if nombre_stages == 1:
        print("Bienvenue !")
    else:
        print("Bon retour parmi nous !")
    print("Profite bien du stage !")
\end{minted}

Que se passe-t-il si on entre 0 ? Et si on entre 1 ? Et si on entre 2 ?

Que faire si l'on a plus de deux cas ? Comme tu l'as vu plus haut, on
peut imbriquer les structures if else. Ce sera aussi le cas pour toutes
les autres structures que nous verrons et c'est très souvent utilisé.

Mais dans le cas des conditionnelles, on a un autre moyen plus facile et
plus lisible. Voici la syntaxe :

\begin{minted}[autogobble]{python}
if condition1:
    instructions du cas 1
elif condition2:
    instructions du cas 2
elif condition3:
    instructions du cas 3
else:
    instructions dans le reste des cas
\end{minted}

On peut mettre autant de elif que l'on veut. elif est la contraction de
else if, sinon si

\hypertarget{exercice-8}{%
\subsection{Exercice :}\label{exercice-8}}

Réécris le programme précédent avec des \mintinline[]{text}{elif}.

\hypertarget{exercice-9}{%
\subsection{Exercice :}\label{exercice-9}}

Connais-tu le concours Prologin ? C'est un concours de programmation
auquel on peut participer jusqu'à l'année de ses 20 ans. On voudrait
écrire un programme qui demande à l'utilisateur ou l'utilisatrice son
année de naissance, et dit en fonction :

\begin{itemize}
\item
  ``Tu peux participer au concours Prologin.''
\item
  ``C'est la dernière année où tu peux participer au concours
  Prologin.''
\item
  ``Tu ne peux plus participer au concours Prologin.'' Écris-le en
  utilisant la structure elif. Pour cette année, les personnes nées en
  2001 ou après peuvent participer.
\end{itemize}

Parfois, nous aurons besoin de combiner des conditions. Par exemple,
nous voudrons écrire ``condition1 ou condition2''.

Pour ce faire, en python, on utilise les mots-clefs suivants :

\begin{itemize}
\item
  \mintinline[]{text}{or} pour OU
\item
  \mintinline[]{text}{and} pour ET
\item
  \mintinline[]{text}{not} pour NON. Cela permet d'exprimer l'inverse
  d'une proposition. Par exemple, \mintinline[]{text}{not x>0} équivaut
  à \mintinline[]{text}{x <= 0}.
\end{itemize}

\hypertarget{exercice-10}{%
\subsection{Exercice :}\label{exercice-10}}

Traduis les phrases suivantes avec des conditions en python.

\begin{enumerate}
\def\labelenumi{\arabic{enumi}.}
\item
  n est positif ou n = 5
\item
  a différent de 7 et a différent de 9
\item
  x est strictement compris entre 12 et 19
\end{enumerate}

\hypertarget{la-boucle-for}{%
\section{La boucle for}\label{la-boucle-for}}

Nous allons voir ici un moyen de demander à notre ordinateur de répéter
plusieurs fois un même bloc d'instructions. À nouveau, notre programme
ne va pas exécuter les instructions une par une. Il y a un bloc de
lignes qu'il va exécuter dans l'ordre, puis exécuter à nouveau à partir
de la première, un certain nombre de fois. C'est pour ça qu'on parle de
boucle.

\hypertarget{exercice-11}{%
\subsection{Exercice :}\label{exercice-11}}

\begin{minted}[autogobble]{python}
for i in range(5) :
    print("Donnez le nom d’un joueur ou d’une joueuse")
    nom = input()
    print("Nous avons un joueur ou une joueuse du nom de " + nom)
\end{minted}

\begin{enumerate}
\def\labelenumi{\arabic{enumi}.}
\tightlist
\item
  Que fait ce programme ?
\end{enumerate}

\begin{quote}
Comme tu peux le constater, on retrouve la construction de la
conditionnelle avec les~: en fin de ligne et les tabulations. Ce qu'il
faut retenir de la boucle for, c'est que for i in range(n) permet
d'exécuter un morceau de code n fois. On appelle une exécution du bloc
de code indenté sous le for un tour de boucle. Ici, la première fois
qu'on demande le nom, c'est le premier tour de boucle, la deuxième fois,
c'est le deuxième tour de boucle, etc.
\end{quote}

\begin{enumerate}
\def\labelenumi{\arabic{enumi}.}
\setcounter{enumi}{1}
\tightlist
\item
  Ce programme fonctionne si on a exatement 5 joueurs ou joueuses. Mais
  si le nombre de joueurs ou joueuses change à chaque partie ? On ne va
  pas réécrire un programme à chaque fois !
\end{enumerate}

Écris un programme qui demande le nombre de joueurs ou joueuses à
l'utilisateur puis qui exécute la boucle un nombre adapté de fois.
Indice : Rappelle-toi que dans le range, on indique un nombre, mais
qu'un nombre peut être stocké dans une variable\ldots{}

Tu remarques sans doute qu'on préférerait demander ``Quel est le nom de
la joueuse 1 ?'' puis ``Quel est le nom de la joueuse 2 ?'' etc. Ce
serait plus clair.

Eh bien, savoir à quel tour de boucle on se trouve, c'est quelque chose
qui sera souvent utile. C'est pour cette raison qu'on utilise un i dans
\mintinline[]{text}{for i in range(…)}. i est une variable comme
n'importe quelle autre, qu'on peut manipuler, afficher, etc.

Attention : c'est généralement une mauvaise idée d'écrire
\mintinline[]{text}{i =} dans le corps de la boucle. i change
naturellement de valeur à chaque tour de boucle (il augmente de 1).
Changer la valeur de i n'est généralement pas la bonne solution à un
problème.

Note : Nous avons tout écrit avec un i, mais en fait, on peut remplacer
i par n'importe quel nom de variable. Essaie avec j, ou rang, par
exemple. On appelle cette variable la variable de boucle.

\hypertarget{exercice-12}{%
\subsection{Exercice :}\label{exercice-12}}

On veut modifier le programme de l'exercice précédent pour qu'il demande
et affiche le numéro de la joueuse. On propose le programme suivant :

\begin{minted}[autogobble]{python}
print("Combien y a-t-il de joueurs ou joueuses ?")
nombre_joueurs = int(input())
for i in range(nombre_joueurs):
        print("Donnez le nom du joueur ou de la joueuse " + str(i))
        nom = input()
        print("Le joueur ou la joueuse " + str(i) + " s’appelle" + nom)
\end{minted}

\begin{enumerate}
\def\labelenumi{\arabic{enumi}.}
\tightlist
\item
  Essaie ce programme. Est-ce qu'il donne le résultat escompté ?
\end{enumerate}

\begin{quote}
On voit qu'avec range(n), la variable de boucle (i ici) varie entre 0 et
n-1. La boucle est donc bien exécutée n fois, mais i ne varie pas entre
1 et n comme on pourrait s'y attendre. C'est une source courrante
d'erreurs.
\end{quote}

\begin{enumerate}
\def\labelenumi{\arabic{enumi}.}
\setcounter{enumi}{1}
\tightlist
\item
  Modifie le programme pour qu'il ait le comportement attendu (on veut
  que les joueurs ou joueuses soient numérotées de 1 à nombre\_joueurs
  et non de 0 à nombre\_joueurs-1).
\end{enumerate}

\hypertarget{exercice-13}{%
\subsection{Exercice :}\label{exercice-13}}

Écris un programme qui affiche un compteur de 1 à n, où n est un nombre
entré par l'utilisateur ou l'utilisatrice.

\hypertarget{exercice-14}{%
\subsection{Exercice :}\label{exercice-14}}

Écris un programme qui affiche un compteur de 10 à 0, et termine en
affichant ``C'est terminé !''

\hypertarget{exercice-15}{%
\subsection{Exercice :}\label{exercice-15}}

Deux joueuses jouent au ping-pong. Écris un programme qui leur demande
combien de parties elles vont faire puis, pour chaque partie, leur
demande leurs scores et dit qui a gagné.

\emph{Indice :} on peut mettre une structure if else dans une boucle
for.

\hypertarget{exercice-16}{%
\subsection{Exercice :}\label{exercice-16}}

Maintenant, ces deux joueuses jouent au tennis, et elles voudraient
afficher un message qui dit à quel set et à quel match elles sont. Par
exemple, s'il y a trois sets de deux matchs chacun, notre programme
affichera :

\begin{minted}[autogobble]{text}
Nous sommes au set 1, match 1
Nous sommes au set 1, match 2
Nous sommes au set 2, match 1
Nous sommes au set 2, match 2
Nous sommes au set 3, match 1
Nous sommes au set 3, match 2
\end{minted}

Écris un programme qui :

\begin{itemize}
\item
  demande à l'utilisateur ou utilisatrice combien il y aura de sets
\item
  demande à l'utilisateur ou utilisatrice combien il y aura de matchs
  par set
\item
  affiche les phrases sur le modèle ci-dessus.
\end{itemize}

\hypertarget{pour-aller-plus-loin-1}{%
\subsection{Pour aller plus loin :}\label{pour-aller-plus-loin-1}}

En fait, \mintinline[]{text}{range} peut prendre plus d'un paramètre. On
peut utiliser \mintinline[]{text}{for i in range(a,b)} pour faire varier
\mintinline[]{text}{i} entre a et b-1. On peut même utiliser
\mintinline[]{text}{for i in range(a,b,p)} pour faire varier
\mintinline[]{text}{i} entre a et b-1 avec un pas de p.~Par exemple,
pour un compte à rebours, on peut prendre \mintinline[]{text}{p = -1} ou
pour compter de deux en deux, \mintinline[]{text}{p = 2}.

\hypertarget{la-boucle-while}{%
\section{La boucle while}\label{la-boucle-while}}

La boucle while est une structure qui ressemble à la boucle for. Comme
avant, on va exécuter des instructions en boucle, d'où le nom de la
structure.

Cependant, dans la boucle for, on exécutait la boucle un nombre précis
de fois. Ce nombre pouvait être entré par l'utilisateur ou
l'utilisatrice, mais une fois qu'on commençait la boucle, il ne bougeait
plus.

Or, parfois, on a envie d'exécuter une boucle un nombre de fois qu'on ne
peut pas déterminer à l'avance. C'est à ça que sert la boucle while.
Elle effectue le même bloc d'instructions tant que (while en anglais)
une certaine condition est remplie.

Par exemple, on peut faire un programme qui envoie un message
d'encouragement tant que l'utilisateur ou l'utilisatrice le demande :

\begin{minted}[autogobble]{python}
print("As-tu besoin d’encouragements ? Si oui, tape 1, si non, tape 2.")
reponse = int(input())
while reponse == 1 :
    print("Tu vas y arriver ! J’ai confiance en toi !")
    print("As-tu besoin d’encouragements ? Si oui, tape 1, si non, tape 2.")
    reponse = int(input())
\end{minted}

La syntaxe de la boucle while est la même que celles que nous avons vues
précédemment : le \mintinline[]{text}{:} , des indentations. Retiens
aussi qu'on peut utiliser les structures précédentes (la boucle for et
les conditionnelles) avec la boucle while.

\hypertarget{exercice-17}{%
\subsection{Exercice :}\label{exercice-17}}

On joue à un jeu de hasard. L'utilisateur ou utilisatrice doit deviner
un nombre entre 1 et 10. Tant qu'il ou elle n'a pas trouvé, on lui
redemande de proposer un nombre, et on s'arrête quand il ou elle trouve
le nombre secret. Écris un programme permettant de jouer à ce jeu.

\hypertarget{exercice-18}{%
\subsection{Exercice :}\label{exercice-18}}

Maintenant, on veut que le joueur ou la joueuse devine un nombre secret
entre 1 et 100. Mais comme c'est un peu compliqué, on va lui dire à
chaque essai, s'il a proposé un nombre trop grand ou trop petit. Un
exemple d'échange serait le suivant :

\begin{minted}[autogobble]{text}
Proposez un nombre
40
Trop petit !
50
Trop petit !
83
Trop grand !
74
Bravo, vous avez deviné !
\end{minted}

Écris un programme qui permet de jouer à ce jeu.

Bonus : À ton avis, quelle est la meilleure stratégie pour gagner à ce
jeu ?

\hypertarget{exercice-19}{%
\subsection{Exercice :}\label{exercice-19}}

\emph{Boucle for ou boucle while}

On veut écrire un programme qui affiche les entiers pairs entre 0 et 2n,
où n est un nombre donné par l'utilisateur ou l'utilisatrice.

\begin{enumerate}
\def\labelenumi{\arabic{enumi}.}
\item
  Écris un tel programme en utilisant une boucle for.
\item
  Écris un tel programme en utilisant une boucle while.
\end{enumerate}

\hypertarget{les-fonctions}{%
\section{Les fonctions}\label{les-fonctions}}

Si tu as déjà vu les fonctions en cours de mathématiques, cette partie
devrait te rappeler quelque chose. Sinon, voilà une petite explication
de ce que sont les fonctions en python.

Une fonction est comme une boîte noire. On lui donne un ou plusieurs
objets qu'on appelle les entrées, et elle nous donne en retour un ou
plusieurs objets qu'on appelle sorties et/ou elles font une action
(afficher un message par exemple). Et à l'intérieur, eh bien, on ne sait
pas forcément ce qui se passe. Bien sûr, quand nous écrirons nos propres
fonctions, on saura ce qu'elles font ! Mais tu as déjà utilisé des
fonctions sans le savoir : \mintinline[]{text}{print} et
\mintinline[]{text}{input}. Ces fonctions ne renvoient rien, elles n'ont
pas de sortie. Cela arrive.

\mintinline[]{text}{print} prend en entrée un nombre, ou du texte, et
affiche cette entrée à l'écran.

\mintinline[]{text}{input} ne prend pas d'entrée, et ne renvoie rien,
mais permet de récupérer une valeur entrée par l'utilisatrice.

On sait ce que font ces deux fonctions, mais pas comment elles le font.
C'est ce qu'on veut dire par boîte noire. Beaucoup de lignes de code se
cachent derrière \mintinline[]{text}{print} et
\mintinline[]{text}{input} mais on n'a pas besoin de les connaître.
Savoir quel effet ont ces deux fonctions suffit à les utiliser.

Nous allons écrire une fonction qui prend en entrée un nombre
\mintinline[]{text}{x} et renvoie son double.

\begin{minted}[autogobble]{python}
def double(x):
    resultat = 2*x
    return resultat
\end{minted}

Discutons maintenant de la forme de cette fonction :

\begin{itemize}
\item
  On retrouve les \mintinline[]{text}{:} et les tabulations comme dans
  toutes les autres structures
\item
  Les entrées, ici \mintinline[]{text}{x}, sont indiquées entre
  parenthèses. Si on veut mettre plusieurs entrées, il faut les séparer
  avec des virgules, par exemple : \mintinline[]{text}{f(x,y)}.
\item
  Le mot-clef \mintinline[]{text}{return} sert à indiquer la sortie de
  la fonction. Quand on atteint une ligne avec return, on sort du code
  de la fonction, le reste n'est pas exécuté.
\end{itemize}

Maintenant, si on tape \mintinline[]{text}{print(double(4))} on obtient
:

\begin{minted}[autogobble]{text}
8
\end{minted}

Pour donner une valeur en entrée à une fonction, ici
\mintinline[]{text}{4}, on doit aussi le mettre entre parenthèses, après
le nom de la fonction.

\hypertarget{exercice-20}{%
\subsection{Exercice :}\label{exercice-20}}

Voici une fonction.

\begin{minted}[autogobble]{python}
def f(x):
    resultat = x + 3
    return resultat
\end{minted}

Quel est la valeur de \mintinline[]{text}{f(17)} ? De
\mintinline[]{text}{f(0)} ? Vérifie en écrivant la fonction sur ton
ordinateur.

\hypertarget{exercice-21}{%
\subsection{Exercice :}\label{exercice-21}}

Voici une fonction.

\begin{minted}[autogobble]{python}
def f(x):
    if x <= 5:
        resultat = x * x
    else :
        resultat = x*5
    return resultat
\end{minted}

Quel est la valeur de \mintinline[]{text}{f(3)} ? De
\mintinline[]{text}{f(5)} ? De \mintinline[]{text}{f(7)} ? Vérifie en
écrivant la fonction sur ton ordinateur.

\hypertarget{exercice-22}{%
\subsection{Exercice :}\label{exercice-22}}

Écris une fonction qui prend en entrée un nombre et renvoie le triple de
ce nombre.

\begin{quote}
À quoi sert une fonction ? Pas seulement à faire des calculs ! Avoir des
fonctions, c'est pouvoir diviser son programme en petits morceaux
autonomes. Pour programmer avec moins d'erreurs, on va souvent découper
notre programme en fonctions. Nous allons voir des exemples dans le
prochain exercice, en reprenant un de nos programmes précédents pour
l'écrire avec des fonctions.
\end{quote}

\hypertarget{exercice-23}{%
\subsection{Exercice :}\label{exercice-23}}

Reprenons le jeu où on doit deviner un nombre secret entre 1 et 100.

\begin{enumerate}
\def\labelenumi{\arabic{enumi}.}
\tightlist
\item
  Écris une fonction \mintinline[]{text}{tour(secret, nombre)} qui
  affiche selon le cas ``Trop petit !'', ``Trop grand !'' ou ``Bravo, tu
  as gagné !''
\end{enumerate}

\emph{Indice :} cette fonction ne contiendra pas de
\mintinline[]{text}{return}.

\begin{enumerate}
\def\labelenumi{\arabic{enumi}.}
\setcounter{enumi}{1}
\tightlist
\item
  Écris maintenant un programme qui permet de jouer au jeu en utilisant
  ta fonction.
\end{enumerate}

\hypertarget{probluxe8me-le-jeu-de-nim}{%
\subsection{Problème : Le jeu de Nim}\label{probluxe8me-le-jeu-de-nim}}

Nous allons coder un programme qui permet à deux joueurs ou joueuses de
s'affronter au jeu de Nim. Les règles sont très simples : on commence
avec vingt bâtons, et chacun son tour, on en retire un, deux, ou trois
de la pile. La personne qui prend le dernier bâton gagne. Nous voudrions
que le résultat du programme ressemble à ça :

\begin{minted}[autogobble]{text}
Il y a 20 bâtons.
Joueur ou joueuse 1, c'est à toi ! Combien de bâtons veux-tu enlever ?
3
Il y a 17 bâtons.
Joueur ou joueuse 2, c'est à toi ! Combien de bâtons veux-tu enlever ?
2
Il y a 15 bâtons.
Joueur ou joueuse 1, c'est à toi ! Combien de bâtons veux-tu enlever ?
3
Il y a 12 bâtons.
Joueur ou joueuse 2, c'est à toi ! Combien de bâtons veux-tu enlever ?
2
Il y a 10 bâtons.
Joueur ou joueuse 1, c'est à toi ! Combien de bâtons veux-tu enlever ?
1
Il y a 9 bâtons.
Joueur ou joueuse 2, c'est à toi ! Combien de bâtons veux-tu enlever ?
1
Il y a 8 bâtons.
Joueur ou joueuse 1, c'est à toi ! Combien de bâtons veux-tu enlever ?
3
Il y a 5 bâtons.
Joueur ou joueuse 2, c'est à toi ! Combien de bâtons veux-tu enlever ?
3
Il y a 2 bâtons.
Joueur ou joueuse 1, c'est à toi ! Combien de bâtons veux-tu enlever ?
2
Il ne reste plus de bâtons. Joueur ou joueuse 1, tu remportes la partie !
\end{minted}

\begin{enumerate}
\def\labelenumi{\arabic{enumi}.}
\tightlist
\item
  Crée une fonction \mintinline[]{text}{tour(numero_joueuse,nb_batons)}
  qui permet d'afficher :
\end{enumerate}

\begin{minted}[autogobble]{text}
Il y a nb_batons batons.
Joueur ou joueuse numero_joueuse, c'est à toi ! Combien de bâtons veux-tu enlever ?
\end{minted}

et qui renvoie le nouveau nombre de bâtons.

\begin{enumerate}
\def\labelenumi{\arabic{enumi}.}
\setcounter{enumi}{1}
\tightlist
\item
  Avec cette fonction, écris un programme permettant de jouer au jeu de
  Nim.
\end{enumerate}

\emph{Indice :} Il y a un nouveau tour tant que le nombre de bâtons
n'est pas égal à 0.

\begin{enumerate}
\def\labelenumi{\arabic{enumi}.}
\setcounter{enumi}{2}
\tightlist
\item
  Crée une fonction \mintinline[]{text}{jeu_de_Nim} avec ton précédent
  programme qui permet de jouer à une partie du jeu. Ensuite, crée un
  programme qui permet de jouer tant que le joueur ou la joueuse le
  désire.
\end{enumerate}

Bonus : Il y a une manière de gagner à coup sûr à ce jeu si on est le
premier joueur ou la première joueuse. Essaie de la trouver.

\end{document}
